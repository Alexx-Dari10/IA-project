\documentclass[twoside]{article}

\usepackage{lipsum}
\usepackage[none]{hyphenat} 

\usepackage[sc]{mathpazo} 
\usepackage[T1]{fontenc} 
\linespread{1.05}
\usepackage{microtype}

\usepackage[hmarginratio=1:1,top=32mm,columnsep=20pt]{geometry}
\usepackage{multicol} 
\usepackage[hang, small,labelfont=bf,up,textfont=it,up]{caption} 
\usepackage{booktabs} 
\usepackage{float} 
\usepackage{hyperref}
\usepackage{amsmath}

\usepackage{lettrine} 
\usepackage{paralist}

\usepackage{abstract} 
\renewcommand{\abstractnamefont}{\normalfont\bfseries}
\renewcommand{\abstracttextfont}{\normalfont\small\itshape} 

\usepackage{titlesec} 
\renewcommand\thesection{\Roman{section}} 
\renewcommand\thesubsection{\Roman{subsection}} 
\titleformat{\section}[block]{\large\scshape\centering}{\thesection.}{1em}{} 
\titleformat{\subsection}[block]{\large}{\thesubsection.}{1em}{}

\usepackage{fancyhdr} 
\pagestyle{fancy} 
\fancyhead{} 
\fancyfoot{}

\fancyhead[C]{ IA $\bullet$ Proyecto Final $\bullet$ C-511}
\fancyfoot[RO,LE]{\thepage}

\title{\vspace{-0.5cm}\fontsize{20pt}{10pt}\selectfont\textbf{Reconocimiento de d\'igitos num\'ericos escritos a mano}}

\author{
\large
\textsc{\vspace{-2cm} Alejandro Campos, Darian Dominguez}\\[3.5cm]
\normalsize Facultad de Matem\'atica y Computaci\'on \\
\normalsize Universidad de la Habana \\
\normalsize 2021 \\[1cm]
\vspace{-5mm}
}
\date{}


\usepackage{graphicx}
\begin{document}

\maketitle

\thispagestyle{fancy} 

\begin{center}
\textbf{Abstract}
\end{center}
\noindent \textit{ El Reconocimiento \'Optico de Caracteres es una l\'inea de investigaci\'on dentro del procesamiento de im\'agenes para la que se han desarrollado muchas t\'ecnicas y metodolog\'ias. Su objetivo principal consiste en identificar un caracter a partir de una imagen digitalizada que se representa como un conjunto de p\'ixeles. En este trabajo realizaremos reconocimiento de d\'igitos num\'ericos escritos a mano, utilizando m\'etodos de tratamiento de imagen y modelos de clasificaci\'on de deep learning.}\\[0.5cm]

\end{document}